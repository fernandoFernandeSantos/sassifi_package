\section{Discussion}

SASSI has some clear shortcomings.  First, as we mentioned in the
introduction, it is not a binary instrumenter, so source code is
required.  Furthermore, the additional complexity of the build process
has proven to be fairly difficult for some users.  A true binary
instrumentation approach like that of \texttt{cuda-memcheck} would
make the process seamless.  On the other hand, there are some
advantages: we are investigating porting the tool to the driver, which
would allow us to handle graphics applications (which are
JITted)\footnote{For this case, it does not make sense to have the
  driver compile the application, then disassemble it, patch it, then
  re-apply.}; in
addition, SASSI can gather potentially useful properties that the
compiler knows about such as register liveness, operand datatypes,
basic block boundaries, etc.  Related to this, the instrumentation
code should be much less intrusive than that of binary patching.
