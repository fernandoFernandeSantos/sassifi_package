\section{Instrumenting an application}

This section describes how to instrument an application.  The
high-level instrumentation flow is the same as the standard
compilation flow shown in Figure~\ref{fig:sassi-comp-flow}.  We use a
SASSI-modified version of \texttt{ptxas} to instrument all of the modules
in the GPGPU application.  In addition, we define \emph{what}
instrumentation we are going to perform in a separate CUDA library,
compile it separately, and link it in with the other modules in the
program.  There are some important requirements we impose on both the
instrumented code and instrumentation libraries.  We describe these
requirements in Section~\ref{sec:important-required}.

A SASSI-modified version of \texttt{ptxas} accepts a few optional
arguments not present in the default \texttt{ptxas}.  All
instrumentation intent is conveyed to SASSI via command-line arguments
to \texttt{ptxas}.  Section~\ref{sec:command-line-args} lists all of
the available command-line options for SASSI.  Note that SASSI's user
interface is currently simple, can be limiting, and is subject to
change. In general, SASSI requires two pieces of information to
instrument an application.  First, we need to tell SASSI \emph{where}
it should insert instrumentation; and second, we need to tell SASSI
\emph{what} information it should extract for each instrumentation
site.

\subsection{Instrumentation sites (the \emph{where})}

We will first describe the knobs that SASSI exposes to convey
\emph{where} it should insert instrumentation.  SASSI has a few flags
for guiding instrumentation sites, but the two most common are:\\
\begin{center}
\texttt{--sassi-inst-before=}``$\langle$ list of instruction
\emph{classes} from Table~\ref{tab:inst-before}$\rangle$'' \\

\texttt{--sassi-inst-after=}``$\langle$ list of instruction
\emph{classes} from Table~\ref{tab:inst-after}$\rangle$'' \\
\end{center}

As the name implies \texttt{--sassi-inst-before} inserts
instrumentation code before certain instructions, and
\texttt{--sassi-inst-after} inserts code after
instructions. {\hl{Please read section}~\ref{sec:after-pitfalls} \hl{for restrictions and pitfalls when inserting instrumentation after instructions.}}

The two
flags take a comma-separated list of instrumentation sites.

\begin{table}[h]
\centering
\begin{tabular}{|l|l|}
\hline
Flag & Description \\
\hline
all & Inserts instrumentation before \emph{all} instructions. \\
calls & Inserts instrumentation before function calls. \\
cond-branches & Inserts instrumentation before conditional branches. \\
memory & Inserts instrumentation before memory operations.  SASSI does {\bf not} consider ``load constant''' to be memory operations. \\
reg-reads & Inserts instrumentation before instructions that read
registers. \\
reg-writes & Inserts instrumentation before instructions that write
registers. \\
\hline
\end{tabular}
\caption{Possible instrumentation sites for \texttt{--sassi-inst-before}.}
\label{tab:inst-before}
\end{table}

\begin{table}[h]
\centering
\begin{tabular}{|l|l|}
\hline
Flag & Description \\
\hline
calls & Inserts instrumentation after function calls. \\
memory & Inserts instrumentation after memory operations. SASSI does {\bf not} consider ``load constant''' to be memory operations. \\
reg-writes & Inserts instrumentation after instructions that write
registers. \\
\hline
\end{tabular}
\caption{Possible instrumentation sites for \texttt{--sassi-inst-after}.}
\label{tab:inst-after}
\end{table}

In addition to the \texttt{--sassi-inst-before/after} flags, we can
have SASSI insert instrumentation at kernel entry points with the
flag,\texttt{--sassi-kernel-entry}.  There are a couple of other
instrumentation points that we will expose shortly in a future release.

As an example, consider the case where we want to insert
instrumentation before instructions that touch memory \emph{or}
instructions that write any registers, we would provide \texttt{ptxas}
the argument,\\ 
\begin{center}
\texttt{--sassi-inst-before=}``\texttt{memory,reg-writes}''.
\end{center}

If you pass \texttt{--sassi-inst-before} or
\texttt{--sassi-inst-after} invalid arguments, SASSI will abort the
compilation with an error.  The \texttt{--sassi-inst-after} option
does not allow for instrumentation after any class of instructions
that includes branches. This simplifies code generation and user
expectations.  Note, as Section~\ref{sec:example2} shows, it is still
possible to determine where a branch \emph{will} go using SASSI.

Let's compile an application and instruct SASSI to inject
instrumentation before all memory operations:

\begin{lstlisting}[style=BashInputStyle]
nvcc gaussian.cu -gencode arch=compute_50,code=sm_50 \
  -Xptxas --sassi-inst-before=''memory'' -O3  -c -rdc=true -o gaussian.o  -std=c++11
\end{lstlisting}

With this command, SASSI will spit out the following message:
\begin{verbatim}
******************************************************************************
*
*                       SASSI Instrumentation Details
*
*  For the settings you passed in, you'll need to make sure that you have
*  an instrumentation library with the following properties:
*  - It MUST BE compiled using only 16 registers!! To accomplish this
*    simply compile your library with the nvcc flag, --maxrregcount=16
*  - It must define the following functions:
*      __device__ void sassi_before_handler(SASSIBeforeParams*)
*
******************************************************************************
\end{verbatim}

This message is a good segue to the next topic--- \emph{what}
instrumentation gets inserted.

\subsection{Instrumentation arguments (the \emph{what})}

The instrumentation that SASSI injects is essentially a function call
to a CUDA function that the user separately defines.  So the SASSI
commands we used last will insert a call to the following function
before every memory operation:
\begin{center}
\begin{lstlisting}
__device__ void sassi_before_handler(SASSIBeforeParams*);
\end{lstlisting}
\end{center}

You can, with a couple of minor limitations that we discuss in
Section~\ref{sec:what-can-go},  
put whatever CUDA computation you want in
\texttt{sassi\_before\_handler}, and it will execute before every
memory operation executes.

SASSI assumes the user wants access to
\emph{baseline} information about each instrumented instruction, so it
bundles up basic information about the instrumented instruction,
places it in a C++ object of type \texttt{SASSIBeforeParams}, and
passes along a pointer to the object as an argument to the handler.
The argument type is shown in Figure~\ref{fig:before-args}.  It allows
us to test whether the instruction will execute (which it might not
because NVIDIA's architectures feature predicated execution), get its
opcode, get its virtual PC, etc.

\begin{figure*}
\center
\begin{lstlisting}
class SASSIBeforeParams : public SASSICoreParams {
public:
  ///////////////////////////////////////////////////////////////////////////////
  // Returns true if the instruction will execute.
  ///////////////////////////////////////////////////////////////////////////////
  __device__ bool GetInstrWillExecute() const;

 //////////////////////////////////////////////////////////////////////////////
  // Gets the opcode of the instruction.
  //////////////////////////////////////////////////////////////////////////////
  __device__ SASSIInstrOpcodes GetOpcode() const;

  //////////////////////////////////////////////////////////////////////////////
  // Returns the offset of the instruction from the function's entry.
  ///////////////////////////////////////////////////////////////////////////////
  __device__ uint32_t GetInsOffset() const;

  //////////////////////////////////////////////////////////////////////////////
  // Returns the name of the kernel/function that contains this instruction.
  ///////////////////////////////////////////////////////////////////////////////
  __device__ const char *GetFnName() const;

  //////////////////////////////////////////////////////////////////////////////
  // Gets a probably unique virtual PC, which is repeatable between runs.
  ///////////////////////////////////////////////////////////////////////////////
  __device__ uint64_t GetPUPC() const;

  //////////////////////////////////////////////////////////////////////////////
  // Returns the SHA1 hash of the function name.
  ///////////////////////////////////////////////////////////////////////////////
  __device__ const uint32_t *GetFnNameSHA1() const;

  ///////////////////////////////////////////////////////////////////////////////
  // Various ways to query whether the instruction is a memory operation.
  ///////////////////////////////////////////////////////////////////////////////
  __device__ bool IsMem() const;
  __device__ bool IsMemRead() const;
  __device__ bool IsMemWrite() const;
  __device__ bool IsSpillOrFill() const;
  __device__ bool IsSurfaceMemory() const;

  ///////////////////////////////////////////////////////////////////////////////
  // Returns true if the instruction type is a control transfer operation.
  ///////////////////////////////////////////////////////////////////////////////
  __device__ bool IsControlXfer() const;
  __device__ bool IsConditionalControlXfer() const;

  // Other methods follow...
};
\end{lstlisting}
\caption{The baseline parameter that's passed in to an instrumentation
  handler allows the instrumentation handler to query the above methods.} 
\label{fig:before-args}
\end{figure*}

While we can query basic characteristics with
\texttt{SASSIBeforeParams} objects, this information is clearly
insufficient for many studies. We can optionally specify 
additional information to extract for each instruction.  
 If the user has specified that instrumentation should be inserted
before some set of instructions (using the
\texttt{--sassi-inst-before} flag), then we can optionally
specify additional arguments to send to the handler by using the
\texttt{--sassi-before-args} flag.  This flag also takes a
comma-separated list of values.  Currently, allowed values for
\texttt{--sassi-before-args} are `cond-branch-info', `mem-info',
and `reg-info'.  Table~\ref{tab:before-args} summarizes the
functions of these values.  Similarly, Table~\ref{tab:after-args}
summarizes the allowable options for ``after'' instrumentation. 

\begin{table}
\center
\begin{tabular}{|l|L{3.4in}|}
\hline Flag & Description \\ \hline cond-branch-info & Extracts
information about conditional branches and passes it to the
instrumentation handler.  \emph{For instrumentation sites that do not
  correspond to a conditional branch instruction, the instrumentation
  handler will be passed a NULL pointer}. See
\texttt{\$SASSI\_HOME/include/sassi/sassi-branch.hpp} for a description
of the object type passed to the handler. \\ \hline mem-info &
Extracts information about memory operations and passes it to the
instrumentation handler.  \emph{For instrumentation sites that do not
  correspond to memory instructions, the instrumentation handler will
  be passed a NULL pointer}.  See
\texttt{\$SASSI\_HOME/include/sassi/sassi-memory.hpp} for a description
of the object type passed to the handler. \\ \hline reg-info &
Extracts information about the instruction's register usage.  The
pointer passed to the instrumentation handler should \emph{always}
point to a valid structure.  See
\texttt{\$SASSI\_HOME/include/sassi/sassi-regs.hpp} for a description of
the object type passed to the handler. \\ \hline
\end{tabular}
\caption{If the \texttt{--sassi-inst-before} flag is used, the
  user may also consider specifying additional information, using the
  \texttt{--sassi-before-args} flag, about the instrumentation site
  that will be passed to the handler.}
\label{tab:before-args}
\end{table}

\begin{table}
\center
\begin{tabular}{|l|L{3.4in}|}
\hline Flag & Description \\ \hline 
mem-info &
Extracts information about memory operations and passes it to the
instrumentation handler.  \emph{For instrumentation sites that do not
  correspond to memory instructions, the instrumentation handler will
  be passed a NULL pointer}.  See
\texttt{\$SASSI\_HOME/include/sassi/sassi-memory.hpp} for a description
of the object type passed to the handler. \\ \hline reg-info &
Extracts information about the instruction's register usage.  The
pointer passed to the instrumentation handler should \emph{always}
point to a valid structure.  See
\texttt{\$SASSI\_HOME/include/sassi/sassi-regs.hpp} for a description of
the object type passed to the handler. \\ \hline
\end{tabular}
\caption{If the \texttt{--sassi-inst-after} flag is used, the
  user may also consider specifying additional information, using the
  \texttt{--sassi-after-args} flag, about the instrumentation site
  that will be passed to the handler.}
\label{tab:after-args}
\end{table}

When you specify additional arguments to be passed, SASSI will tell
you the order it in which it will pass the arguments to the handler.
It is important that you link in a handler that matches the type
signature SASSI demands, or your link step will fail.

\subsection{After Instrumentation: Use Caution}
\label{sec:after-pitfalls}

{\hl{Instrumenting after instructions in general requires extra care.}}
While SASSI makes it possible to add instrumentation after most
instructions, some things are awkward or misleading. SASSI does not
preserve state across instrumented instructions. When passing
parameters to ``after'' instrumentation handlers, use caution because
the state may have changed. For instance, consider this example:
\begin{verbatim}
LD R4 [R4+8]
\end{verbatim}
where we are loading a value from the address in \texttt{R4} plus
eight, and then placing the value loaded back in \texttt{R4}. SASSI
allows you to add instrumentation after memory operations, and
furthermore, it allows you to check the address of the memory
operation. In this case, however, the address will be wrong, because
the value used to compute the address was clobbered. As another
example, it is possible to pass register information to an ``after''
instrumentation handler. You can query which registers are sources and
destinations, and you can even inspect their values. However, the
values are those after the instruction has executed. In the case of
the above example, \texttt{R4} is both a source and a destination. If we check
to see what the value in \texttt{R4} is, it will be the value after the
instruction has executed. A soon-to-be-released version will restrict
what is possible in ``after'' instrumentation to reduce this
confusion.
