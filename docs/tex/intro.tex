\section{Introduction}
\label{sec:intro}

SASSIFI provides an automated framework to perform error injection campaigns
for GPU application resilience evaluation.  SASSIFI builds on top of SASSI,
which is a low-level assembly-language instrumentation tool  that provides the
ability to instrument instructions in the low-level GPU assembly language
(SASS)~\cite{SASSI_ISCA}.

SASSIFI can be used to perform many types of resilience studies. Some examples
are:

\begin{enumerate}

	\item What is the probability that a particle-strike on the
		register file while an application is running will produce an
		SDC? Answering this question allows us to quantify the benefit
		of adding ECC to the register file.  If the SDC probability is
		low, the energy cost of enabling ECC may not be justified. 
		This is a commonly used architecture-level error model.

	\item What is the probability that a bit-flip in the destination
		register of an executing instruction will result in an SDC?  This study
		aims to quantify the effect of bit-flips in low-level unprotected state at
		the application-level by injecting the manifestation at the
		architecture-level, a commonly studied topic in academia.

	\item What instruction types are likely to produce more SDCs when
		subjected to errors in destination registers? This study can provide 
		insights into what to protect while employing selective
		instruction-level protection/duplication schemes.

	\item How do SDC probabilities change when different architecture-level
		states are subjected to errors (e.g., injecting errors into
		values vs.\ addresses of executing instruction)?  How do the
		results change if we inject different bit-flip patterns (e.g.,
		injecting single vs.\ double bit flips)?  Addressing these
		questions provide insight into the accuracy of the commonly
		used error model (single-bit flips in values) in evaluating
		resilience of an application from bit-flips in low-level state
		(e.g., flip-flops) and what to consider for future studies.

\end{enumerate}

For these studies, we want to inject errors using different error models. Since
SASSIFI injects errors through instrumentation handlers, we created three modes
of operation which require instrumentation handlers to be inserted before,
after, and both before and after the instruction, respectively. 

For studies such as (1), we designed a mode called the {\it RF mode} where we
inject bit-flips in the Register File (RF), randomly spread across time and
space (among allocated registers).  For studies such as (2) and (3), we
designed a mode that injects errors into the destination register values by
instrumenting instructions after they are executed. We call this mode {\it IOV}
(Instruction Output Value). Finally, we desinged the third mode called the {\it
IOA} (Instruction Output Address) mode to inject errors into destination
register indices and store addresses. We can use the IOA and IOV modes to
perform the fourth study. 

For more details about how SASSIFI can be used to conduct resilience evaluation
studies and some results, please see the SASSIFI ISPASS 2017
paper~\cite{SASSIFI_ISPASS}.  The resilience case study in the SASSI ISCA 2015
paper~\cite{SASSI_ISCA} and SASSIFI SELSE 2015
presentation~\cite{SASSIFI_SELSE} also offer some details and results. 
Portions of the documentation are from our IPASS 2017 paper, which is
copyrighted by IEEE.


\section{Where can SASSIFI inject errors?}
\label{sec:where}

For the IOA-mode (instruction output-level injections), SASSIFI can inject
errors in the outputs of randomly selected instructions. SASSIFI allows us to
select different types of instructions to study how error in them will
propagate to the application output. As of now (3/10/2017), SASSIFI supports
selecting the following instruction groups. 
\begin{itemize}
\item Instructions that write to general purpose registers (GPR) 
\item Instructions that write to condition code (CC) 
\item Instructions that write to a predicate register (PR) 
\item Store instruction (ST)
\item Integer add and multiply instructions (IADD-IMAD-OP)
\item Single precision floating point add and multiply instructions (FADD-FMUL-OP)
\item Double precision floating point add and multiply instructions (DFADD-DFMUL-OP)
\item Integer fused multiply and add (MAD) instructions (MAD-OP)
\item Single precision floating point fused multiply and add (FMA) instructions  (FMA-OP)
\item Double precision floating point fused multiply and add (DFMA) instructions  (DFMA-OP)
\item Instructions that compare source registers and set a predicate register (SETP-OP)
\item Loads from shared memory (LDS-OP)
\item Load instructions, excluding LDS instructions (LD-OP)
\end{itemize}

SASSIFI can be extended to include custom instruction groups. Follow
instructions in Section~\ref{sec:new_inst_group} to create new instruction
groups. Details about the current instruction grouping, i.e., which SASS
instructions are included in different groups, can be found in
\$SASSIFI\_HOME/err\_injector/error\_injector.h.

In the IOA mode, SASSIFI supports selecting the following two instruction
groups -- GPR and ST. At these instructions, SASSIFI injects errors into the
address (register index or store address) based on the following defined
bit-flip models. 


For the RF mode (injections to measure RF AVF), SASSIFI selects a dynamic
instruction randomly from a program and injects an error in a randomly selected
register among the allocated registers.  Results obtained from these injections
will quantify the probability with which a particle strike in an allocated
register can manifest in the application output.  These results need to be
further derated by the fraction of physical registers that are unallocated to
obtain AVF of the register file for a specific device. 

We can obtain the average number of allocated physical registers per kernel by
multiplying the number of statically allocated registers per thread with the
average number of active threads (number of active warps $\times$ 32) in an SM.
We divide this value with the total number of physical registers in the SM to
obtain the fraction of allocated registers.  We can obtain the average number
of active warps from the {\it achieved\_occupancy} metric as printed by the
{\it nvprof} tool. We can obtain the number of allocated registers during the
compilation step using {\it -Xptxas -v} option. We then use these per kernel
dertaing factors and weigh them with the relative kernel runtimes for each
application to obtain a per application derating factor.

\section{What errors can SASSIFI inject?}
\label{sec:what}

For the IOV mode, SASSIFI can inject the error in a destination register based on
the different Bit Flip Models ({\it BFM}).  In the current release, the following BFMs are implemented. 

\begin{enumerate}
\item Single bit-flip: one bit-flip in one register in one thread
\item Double bit-flip: bit-flips in two adjacent bits in one register in one thread
\item Random value: random value in one register in one thread
\item Zero value: zero out the value of one register in one thread
\item Warp wide single bit-flip: one bit-flip in one register in all the threads in a warp
\item Warp wide double bit-flip: bit-flips in two adjacent bits in one register in all the threads in a warp
\item Warp wide random value: random value in one register in all the threads in a warp
\item Warp wide zero value: zero out the value of one register in all the threads in a warp
\end{enumerate}

In the current implementation, we can only inject single bit-flip in one
register in one thread (first bit-flip model) for the CC and PR injections. For
the SETP-OP instruction group, we can inject only single bit-flip and warp
wide single bit-flip (first and fifth bit-flip models, respectively).

For the IOA and RF modes, SASSIFI considers the following two bit-flip models.  
\begin{itemize}
\item Single bit-flip 
\item Double bit-flip 
\end{itemize}

These BFMs can be extended to include different bit-flip pattern. To add a new
bit-flip model err\_injector.h and injector.cu files in err\_injector directory
and common\_params.py and specific\_params.py files in the scripts directory
need to be modified. 

